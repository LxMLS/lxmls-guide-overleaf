\subsection{Desktops vs. Laptops}

If you have decided to use one of our provided desktops, all installation procedures have been carried out. You merely need to go to the \verb+lxmls-toolkit-student+ folder inside your home directory and start working! You may go directly to section \ref{sec:SolvingExercises}. If you wish to use your own laptop, you will need to install Python, the required Python libraries and download the LXMLS code base. It is important that you do this as soon as possible (before the school starts) to avoid unnecessary delays. Please follow the install instructions. 

\subsection{Basic Install and Troubleshooting}

To install, just follow the instructions in our Github repository for the \verb+student+ version of our toolkit 

\begin{itemize}
\item \url{https://github.com/LxMLS/lxmls-toolkit/tree/student#readme}. 
\end{itemize}

The \verb+student+ branch contains the same code as \verb+master+ branch, with some parts deleted, which you must complete in the following exercises. 

The basic install instructions use miniconda. If this is not your tool of choice, you can install it with \texttt{pip} as an alternative. In order to do this, you first need to run

 \texttt{python3 -m venv lxmls\_venv}

where \texttt{lxmls\_venv} is the name (and path) to your virtual environment. Then, you activate the environment with 

\texttt{source lxmls\_venv/bin/activate} 

and afterwards you install jupyter with pip:

\texttt{pip install jupyter}

Note that, after activating the environment with the command \texttt{source}, the name of the environment should be shown between parentheses in the command line. Make sure this is the case before installing jupyter. After this, you can double check the installation worked by running the following command:

\texttt{pip install ipykernel}

if this installation also worked, then you just need to run the following command so your new virtual environment is also shown in the jupyter notebooks as an interpreter of your choice.

\texttt{python -m ipykernel install -{}-user -{}-name lxmls\_venv -{}-display-name ``LxMLS venv"}


\subsection{Deciding on the IDE and interactive shell to use}

An Integrated Development Environment (IDE) includes a text editor and various tools to debug and interpret complex code. 

\textbf{Important:} As the labs progress you will need an IDE, or at least a good editor and knowledge of pdb/ipdb. This will not be obvious the first days since we will be seeing simpler examples.

Easy IDEs to work with Python are PyCharm and Visual Studio Code, but feel free to use the software you feel more comfortable with. PyCharm and other well known IDEs like Spyder are provided with the Anaconda installation.

Aside of an IDE, you will need an interactive command line to run commands. This is very useful to explore variables and functions and quickly debug the exercises. For the most complex exercises you will still need an IDE to modify particular segments of the provided code. As interactive command line we recommend the Jupyter notebook. This also comes installed with Anaconda and is part of the pip-installed packages. The Jupyter notebook is described in the next section. In case you run into problems or you feel uncomfortable with the Jupyter notebook you can use the simpler iPython command line.

\subsection{Jupyter Notebook}

Jupyter is a good choice for writing Python code. It is an interactive computational environment for data science and scientific computing, where you can combine code execution, rich text, mathematics, plots and rich media. The Jupyter Notebook is a web application that allows you to create and share documents, which contains live code, equations, visualizations and explanatory text. It is very popular in the areas of data cleaning and transformation, numerical simulation, statistical modeling, machine learning and so on. It supports more than 40 programming languages, including all those popular ones used in Data Science such as Python, R, and Scala. It can also produce many different types of output such as images, videos, LaTex and JavaScript. More over with its interactive widgets, you can manipulate and visualize data in real time.

\noindent The main features and advantages using the Jupyter Notebook are the
following:

\begin{itemize}

\item In-browser editing for code, with automatic syntax highlighting, indentation, and tab completion/introspection.

\item The ability to execute code from the browser, with the results of computations attached to the code which generated them.

\item Displaying the result of computation using rich media representations, such as HTML, LaTeX, PNG, SVG, etc. For example, publication-quality figures rendered by the matplotlib library, can be included inline.

\item In-browser editing for rich text using the Markdown markup language, which can provide commentary for the code, is not limited to plain text.

\item The ability to easily include mathematical notation within markdown cells using LaTeX, and rendered natively by MathJax.

\end{itemize}

\noindent The basic commands you should know are

\begin{table}[!h]
\begin{center}
\begin{tabular}{|l|l|}
\hline
Esc              & Enter command mode\\
Enter            & Enter edit mode\\
\hline
up/down          & Change between cells\\
Ctrl + Enter     & Runs code on selected cell\\
Shift + Enter    & Runs code on selected cell, jumps to next cell\\
\hline
restart button   & Deletes all variables (useful for troubleshooting)\\ 
\hline
\end{tabular}
\end{center}
\caption{\label{tb::jupyterbasiccommands}Basic Jupyter commands}
\end{table}

\noindent A more detailed user guide can be found here:

\begin{verbatim}
http://jupyter-notebook-beginner-guide.readthedocs.io/en/latest/index.html
\end{verbatim}
